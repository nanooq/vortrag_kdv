\documentclass{beamer}

\usepackage{default}
\usepackage[utf8]{inputenc}
\usepackage[ngerman]{babel}
\usepackage[autostyle,german=quotes]{csquotes}

\usetheme{Warsaw}

\hypersetup{pdfstartview={Fit}} % fits the presentation to the window when first displayed

\title[TKDV] % (optional, only for long titles)
{TKDV}
\subtitle{Die Totalverweigerer \\und ihre Kriegsdienstverweigerungsanträge}
\author[nanooq nanooq] 
% (optional, for multiple authors)
{nanooq nanooq}
\institute[Hackspace Siegen e.~V.] % (optional)
{
	\inst{1}%
	Hackspace Siegen e.~V.\\
	Effertsufer 104\\
	57072 Siegen
}
\date[2015-03-26] % (optional)
{Lightning Talk am Setting Orange, Discord 12, 3181 YOLD (2015-03-26)\\
	\href{https://github.com/nanooq/vortrag_kdv}{https://github.com/nanooq/vortrag\_kdv}}
\subject{Ziviler Ungehorsam}


\AtBeginSection[]
{
	\begin{frame}
		\frametitle{Inhalt}
		\tableofcontents[currentsection]
	\end{frame}
}

\begin{document}
	
	\frame{\titlepage}
	
	\begin{frame}
	  	\frametitle{Einleitung}	  
	  	\begin{itemize}
	  		\item Die \enquote{Totale Kriegsdienstverweigerung (TKDV)}
	  		\item Ziviler Ungehorsam: Totalverweigerung
	  		\item Schritte zum TKDV
		  	\item[optional] Erfahrungsbericht von meinem eigenen TKDV-Prozess
	  	\end{itemize}		
	\end{frame}

	\begin{frame}
	  	\frametitle{Totalverweigerung}
	  	\framesubtitle{Totale Kriegsdienstverweigerung (TKDV)}
	  	\begin{itemize}
	  		\item Konsequente Form der Kriegsdienstverweigerung. 		
	  		\item Verweigerung aller sich aus der \enquote{Wehrpflicht} ergebenden Zwänge und Zwangsdienste. 
	  		% Die Totalverweigerung betrifft Wehrersatzdienste wie Zivildienst.
	  		% Das Problem unserer Zeit ist nicht der fehlende Gehorsam, sondern der fehlende Ungehorsam
	  		%
	  	\end{itemize}
  	\end{frame}
  	
  	\begin{frame}
	  	\frametitle{Grundgesetz schützt nicht vor Kriegsdienst}
	  	\framesubtitle{Totale Kriegsdienstverweigerung (TKDV)}
	  	
	  	Verweigerung jeder öffentlichen Dienstverpflichtung geht über das im deutschen Grundgesetz im Art. 4 gewährleistete Recht auf Kriegsdienstverweigerung aus Gewissensgründen hinaus.

	  	\begin{block}{Art. 4 Abs. 3 GG}
	  		(3) Niemand darf gegen sein Gewissen zum Kriegsdienst mit der Waffe gezwungen werden. Das Nähere regelt ein Bundesgesetz.
	  	\end{block}
  	\end{frame}
	
	\begin{frame}
	  	\frametitle{Straftatbestand Fahnen- und Dienstflucht}
	  	\framesubtitle{Totale Kriegsdienstverweigerung (TKDV)}

	  	Totalverweigerer kommen zwangsläufig mit strafgesetzlichen Normen in Konflikt, denn die Verweigerung von Kriegs- und Ersatzdienst ist ein Straftatbestand (Fahnen-, Dienstflucht) und kann mit bis zu fünf Jahren Gefängnisstrafe bestraft werden:
	  	
	  	\begin{itemize}
	  		\item Kriegsdienst:  Wehrstrafgesetz (WStG), § 16 Fahnenflucht
	  		\item Ersatzdienst: Zivildienstgesetz (ZDG), § 53 Dienstflucht	  		
	  	\end{itemize}
	\end{frame}
		
	\begin{frame}
	  	\frametitle{Wehrstrafgesetz (WStG), § 16 Fahnenflucht}
	  	\framesubtitle{Totale Kriegsdienstverweigerung (TKDV)}
	  	(1) Wer eigenmächtig seine Truppe oder Dienststelle verläßt oder ihr fernbleibt, um sich der Verpflichtung zum Wehrdienst dauernd oder für die Zeit eines bewaffneten Einsatzes zu entziehen oder die Beendigung des Wehrdienstverhältnisses zu erreichen, wird mit Freiheitsstrafe bis zu fünf Jahren bestraft.\\
	  	(2) Der Versuch ist strafbar.\\
	  	(3) Stellt sich der Täter innerhalb eines Monats und ist er bereit, der Verpflichtung zum Wehrdienst nachzukommen, so ist die Strafe Freiheitsstrafe bis zu drei Jahren.\\
	  	(4) Die Vorschriften über den Versuch der Beteiligung nach § 30 Abs. 1 des Strafgesetzbuches gelten für Straftaten nach Absatz 1 entsprechend.
	\end{frame}
	
	\begin{frame}
	  	\frametitle{Zivildienstgesetz (ZDG), § 53 Dienstflucht}
	  	\framesubtitle{Totale Kriegsdienstverweigerung (TKDV)}
	  	(1) Wer eigenmächtig den Zivildienst verlässt oder ihm fernbleibt, um sich der Verpflichtung zum Zivildienst dauernd oder für den Verteidigungsfall zu entziehen oder die Beendigung des Zivildienstverhältnisses zu erreichen, wird mit Freiheitsstrafe bis zu fünf Jahren bestraft.\\
	  	(2) Der Versuch ist strafbar.\\
	  	(3) Stellt sich der Täter innerhalb eines Monats und ist er bereit, der Verpflichtung zum Zivildienst nachzukommen, so ist die Strafe Freiheitsstrafe bis zu drei Jahren.\\
	  	(4) Die Vorschriften über den Versuch der Beteiligung nach § 30 Abs. 1 des Strafgesetzbuches gelten für Straftaten nach Absatz 1 entsprechend.
	\end{frame}
	
	\begin{frame}
	  	\frametitle{Strafgesetzbuch (StGB), § 30 Versuch der Beteiligung}
	  	\framesubtitle{Totale Kriegsdienstverweigerung (TKDV)}
		(1) Wer einen anderen zu bestimmen versucht, ein Verbrechen zu begehen oder zu ihm anzustiften, wird nach den Vorschriften über den Versuch des Verbrechens bestraft. Jedoch ist die Strafe nach § 49 Abs. 1 zu mildern. § 23 Abs. 3 gilt entsprechend.\\
		(2) Ebenso wird bestraft, wer sich bereit erklärt, wer das Erbieten eines anderen annimmt oder wer mit einem anderen verabredet, ein Verbrechen zu begehen oder zu ihm anzustiften.
	\end{frame}
	
	\begin{frame}
		\frametitle{Wehrpflichtgesetz (WPflG), § 3 Inhalt und Dauer der Wehrpflicht, 1/3}
		\framesubtitle{Totale Kriegsdienstverweigerung (TKDV)}
		(1) Die Wehrpflicht wird durch den Wehrdienst oder im Falle des § 1 des Kriegsdienstverweigerungsgesetzes durch den Zivildienst erfüllt. Sie umfasst die Pflicht, sich zu melden, vorzustellen, nach Maßgabe dieses Gesetzes Auskünfte zu erteilen und Unterlagen vorzulegen, sich auf die geistige und körperliche Tauglichkeit und auf die Eignung für die Verwendungen in den Streitkräften untersuchen zu lassen sowie zum Gebrauch im Wehrdienst bestimmte Bekleidungs- und Ausrüstungsstücke zu übernehmen und entsprechend dem Einberufungsbescheid zum Dienstantritt mitzubringen.\\
	\end{frame}
	
	\begin{frame}
		\frametitle{Wehrpflichtgesetz (WPflG), § 3 Inhalt und Dauer der Wehrpflicht, 2/3}
		\framesubtitle{Totale Kriegsdienstverweigerung (TKDV)}
		(2) Männliche Personen haben nach Vollendung des 17. Lebensjahres eine Genehmigung des zuständigen Kreiswehrersatzamtes einzuholen, wenn sie die Bundesrepublik Deutschland länger als drei Monate verlassen wollen, ohne dass die Voraussetzungen des § 1 Abs. 2 bereits vorliegen. Das gleiche gilt, wenn sie über einen genehmigten Zeitraum hinaus außerhalb der Bundesrepublik Deutschland verbleiben wollen oder einen nicht genehmigungspflichtigen Aufenthalt außerhalb der Bundesrepublik Deutschland über drei Monate ausdehnen wollen. Die Genehmigung ist für den Zeitraum zu erteilen, in dem die männliche Person für eine Einberufung zum Wehrdienst nicht heransteht.
	\end{frame}
	
	\begin{frame}
		\frametitle{Wehrpflichtgesetz (WPflG), § 3 Inhalt und Dauer der Wehrpflicht, 3/3}
		\framesubtitle{Totale Kriegsdienstverweigerung (TKDV)}
		 Über diesen Zeitraum hinaus ist sie zu erteilen, soweit die Versagung für die männliche Person eine besondere – im Bereitschafts-, Spannungs- oder Verteidigungsfall eine unzumutbare - Härte bedeuten würde; § 12 Abs. 6 ist entsprechend anzuwenden. Das Bundesministerium der Verteidigung kann Ausnahmen von der Genehmigungspflicht zulassen.\\
		(3) Die Wehrpflicht endet mit Ablauf des Jahres, in dem der Wehrpflichtige das 45. Lebensjahr vollendet.\\		
		(4) Bei Offizieren und Unteroffizieren endet die Wehrpflicht mit Ablauf des Jahres, in dem sie das 60. Lebensjahr vollenden.\\		
		(5) Im Spannungs- und Verteidigungsfall endet die Wehrpflicht mit Ablauf des Jahres, in dem der Wehrpflichtige das 60. Lebensjahr vollendet.
	\end{frame}
	
	\begin{frame}
	  	\frametitle{Bürokratische Konsequenzen der Verurteilung}
	  	\framesubtitle{Totale Kriegsdienstverweigerung (TKDV)}
		\begin{itemize}
			\item Verurteilung steht für 5 Jahre im Bundeszentralregister. 
			\item Darauf haben bestimmte Behörden Zugriff, z.B. Gerichte und Strafverfolgungsbehörden.
			\item Für das \enquote{polizeiliche} Führungszeugnis gibt es kürzere Fristen (Strafmaß n in Monate):
					\[
					f(n)= 
					\begin{cases}
					\text{0 Jahre}, & \text{if } n < 3\\
					\text{3 Jahre}, & \text{if } 3 \leq n \leq 12\\
					\text{5 Jahre}, & \text{sonst}
					\end{cases}
					\]
		\end{itemize}
	\end{frame}
		
	\begin{frame}
		  	\frametitle{Gesellschaftliche Konsequenzen der Verurteilung}
		  	\framesubtitle{Totale Kriegsdienstverweigerung (TKDV)}
		  	\begin{itemize}
		  		\item Gesellschaftlichen Ächtung: Stress mit Behörden, Freunden, Eltern, Arbeitgebern und so weiter.
		  		% Hamse jedient?
		  		\item Verfahren kostet Zeit und Geld und danach droht eine erneute Heranziehung zum Kriegs- oder Ersatzdienst.
		  		% \enquote{Play it again, Sam!}
		  		\item \enquote{Wer totalverweigern will, sollte sich vorher gründlich Gedanken über die Folgen machen - es gehört Überzeugung dazu und die Courage, für diese Überzeugung einzustehen.} [ASFRAB\_TV]
		  		\item Hall of Fame: Stefan Gierkes Dokumentation über Totalverweigerer seit 2007 [ASFRAB\_GS]. 
		  		% Die Totalverweigerung ist oft eine symbolische Handlung, der Totalverweigerer verzichtet bewusst auf andere, legale oder nicht strafrechtlich verfolgte Möglichkeiten, der Wehrpflicht zu entgehen, und nimmt damit ernstzunehmende Konsequenzen strafrechtlicher und gesellschaftlicher Art in Kauf.
		  	\end{itemize}
	\end{frame}
	
	\begin{frame}
		\frametitle{Gründe zu verweigern, 1}
		\framesubtitle{Ziviler Ungehorsam: Totalverweigerung}
		Unterschied zu opportunistischen Verweigerern:
		\begin{itemize}
			\item Grundsätzlicher Protest und Zurückweisung gegen staatlichen Zwangsdienst und staatliche Bevormundung
			% Ein häufiges Motiv zur Totalverweigerung ist die Auffassung, dass der Staat oder der Gesetzgeber nicht das Recht besitzt, Menschen zur Ableistung von Zwangsdiensten zu verpflichten. Bei diesem Argumentationsschema wird grundsätzlich negiert, dass der Staat das Recht besitzt, über die in seinem Machtbereich befindlichen Menschen wie über persönliche Ressourcen oder Besitzgegenstände zu verfügen. Es wird somit also letztlich die Legitimität der Dienstpflichtigengesetze bestritten, ihre Verabschiedung durch das Parlament als ein Akt gesetzgeberischer Kompetenzüberschreitung charakterisiert, also ausgedrückt, dass der Gesetzgeber etwas „zur Pflicht erhoben hat, das zur Pflicht zu erheben ihm nicht zustand“. Diese Haltung ist natürlich umstritten. Man unterschied in der Rechtsgeschichte den tyrannus quo-ad titulum - Usurpator - und den tyrannus quoad exsecutsflnem - den ungerechten Herrscher. Es gab noch viele weitere Streitfragen dieser Art, immer aber stand außerhalb jeder Diskussion, daß mit dem Recht auf Ungehorsam und dem Recht auf Widerstand ein übergesetzliches Recht eines Menschen oder eines Volkes oder einer Repräsentation des Volkes, ein Grund- und Freiheitsrecht gegenüber dem Staat gemeint gewesen ist. An diesem jahrtausendealten Sach- und Rechtsverhalt kann nicht gerüttelt werden.
			% Daher sieht der Totalverweigerer seinen Akt als eine Form des übergesetzlichen Notstands, der sich aus der Differenz zwischen Legalität und Legitimität ergibt.
			\item Recht auf Ungehorsam, Recht auf Widerstand
			\item Religiöse und politische Überzeugungen, die wie bei der normalen Kriegsdienstverweigerung auch mit persönlichen Gewissensgründen untermauert werden. Oft entspringen sie einer anarchistischen oder pazifistischen Grundhaltung, die direkte Gewalt und hierarchischen Strukturen ablehnt (Pazifisticher Anarchismus).
			\item Ersatzdienstleistende werden im Kriegsfall im Falle eines Krieges eingebunden und benachteiligt.
			\item Ersatzdienst bestraft Verweigerer, obwohl die Verweigerung des Militärdienstes eine richtige, vom eigenen Gewissen vorgeschriebene Handlung darstellt.
		\end{itemize}
	\end{frame}
	
	\begin{frame}
		\frametitle{Gründe zu verweigern, 2}
		\framesubtitle{Ziviler Ungehorsam: Totalverweigerung}
		\begin{itemize}
			\item In einzelnen Fällen ist aber auch ein persönlicher oder biografischer Hintergrund ausschlaggebend, der ebenfalls zu einer unbedingten Ablehnung von Befehl und Gehorsam führen kann.
			\item Ausmusterung von Homosexuellen erlaubt nicht homosexuellen Personen den Wehrdienst zu vermeiden. Diese Ungleichbehandlung wird als diskriminierend abgelehnt und hat nicht die Form eines einklagbaren Rechts.
			\item Protest der Beschränkung der Dienstpflicht auf Männer und die Beschränkung der Heranziehungspraxis auf junge Erwachsene.
			\item Protest gegen die als willkürlich und/oder ungerecht empfundene Einberufungspraxis der Einberufungsbehörden
			\item Gegen die als ungebührlich erachtete Kompetenzdelegierung des Gesetzgebers an Verwaltungsbehörden
			% (der Gesetzgeber ermächtigt die Musterungsärzte und Einberufungsbeamten faktisch dazu, durch ihre Entscheidung „tauglich“ oder „untauglich“ und „Einberufung“ oder „Nicht-Einberufung“ darüber zu bestimmen, ob Menschen massive Freiheitsbeschränkungen erleiden müssen oder ob diese ihnen erspart bleiben und ob Menschen überhaupt erst in die Situation kommen, gegen das Gesetz verstoßen zu können).
		\end{itemize}
	\end{frame}
	  	
%		\begin{block}{This is a Block}
%			This is important information
%		\end{block}
%		\begin{alertblock}{This is an Alert block}
%			This is an important alert
%		\end{alertblock}		
%		\begin{exampleblock}{This is an Example block}
%			This is an example 
%		\end{exampleblock}	
	
	\begin{frame}
		\frametitle{Übersicht Schritte zum TKDV}
	  	\framesubtitle{Schritte zum TKDV}
	  	Wehrdienst ist ausgesetzt, wie wird man nun TKDVler? 
	  	\begin{enumerate}
	  		\item Normalen KDV-Antrag stellen.
	  		\item Sich selber über Konsequenzen der Totalverweigerung informieren.
	  		\item Andere über die absolute Großartigkeit der Totalverweigerung informieren.
	  	\end{enumerate}  			
	\end{frame}
	
	\begin{frame}
		\frametitle{KDV-Antrag, ich?}
		\framesubtitle{Schritte zum TKDV}
		Vielleicht bist du eigentlich schon Totalverweigerer und weißt es nicht? 
		\begin{enumerate}
			\item Unterliegst du als Ungedienter nicht der Wehrpflicht?
			\item Bist du eine Frau? 
			\item Bist du nicht Deutscher nach Art. 116 GG?
			\item Bist du ungemustert oder ausgemustert?
			\item Bist du jünger als 17 $ \frac{1}{2} $ Jahre?
			\item Bist du älter als 60 Jahre?
		\end{enumerate}
		Mindestens eine \enquote{Ja}-Antwort bedeutet, dass du ohne weitere Schritte zu unternehmen schon Totalverweigerer bist. Lob und Anerkennung! Kann das so bleiben? % Nicht mustern lassen um KDV stellen zu können.			
	\end{frame}
	
	\begin{frame}
		\frametitle{KDV-Antrag, ich!}
		\framesubtitle{Schritte zum TKDV}
		Die Teile des Antrags:
		\begin{itemize}
			\item KDV-Antrag: \enquote{... hiermit beantrage ich die Anerkennung als Kriegsdienstverweigerer aus Gewissensgründen gemäß Artikel 4 Absatz 3 des Grundgesetzes.}
			\item Tabellarischer Lebenslauf: Laut Gesetz tabellarisch und vollständig, also ohne zeitliche Lücken. Dazu die persönlichen Daten Name, Geburtsort, -jahr und der eigene vollständige Schul-/Ausbildungsweg und die aktuelle Berufsausübung.
			\item KDV-Begründung: Erläutern und darlegen der eigenen Gewissensentscheidung, ihre Entstehung und Bedeutung im eigenen Leben.
			% Organisationen helfen gerne
		\end{itemize}			
	\end{frame}
	
	\begin{frame}
		\frametitle{KDV-Begründung}
		\framesubtitle{Schritte zum TKDV}
		\begin{itemize}
			\item Die Erläuterung der eigenen Gewissensbildung
			\item Auseinandersetzung mit Krieg, Gewalt, Töten
			\item Auseinandersetzung mit Aufgaben des Militärs bzw. der Bundeswehr
			\item Nothilfe- und Notwehrsituationen
		\end{itemize}
	\end{frame}
	
	\begin{frame}
		\frametitle{KDV-Anerkennung und weiter}
		\framesubtitle{Schritte zum TKDV}
		Gratulation, \enquote{Sie sind berechtigt, den Kriegsdienst mit der Waffe zu verweigern}.\\
		Nun informiere Andere über diese ehrliche Form des zivilen Ungehorsams. \\
		Weitere Informationen:
		\begin{itemize}
			\item \href{http://www.dfg-vk.de/}{http://www.dfg-vk.de/}
			\item \href{http://www.dfg-vk.de/thematisches/bundeswehr-abschaffen/}{http://www.dfg-vk.de/thematisches/bundeswehr-abschaffen/}
			\item \href{http://www.kampagne.asfrab.de/}{http://www.kampagne.asfrab.de/}
			\item \href{http://www.bundeswehr-monitoring.de/}{http://www.bundeswehr-monitoring.de/}
			\item \href{http://www.neinzurnato.de/?page_id=6}{http://www.neinzurnato.de/?page\_id=6}
		\end{itemize} 	
	\end{frame}

	\begin{frame}
		\frametitle{Eigener Erfahrungsbericht}
		\framesubtitle{Ist genug Interesse und Zeit?}
		Vielen Dank für eure Aufmerksamkeit!
		% Private Daten schützen, öffentliche Daten nützen
	\end{frame}
	
	\begin{frame}
		\frametitle{Artikelverzeichnis}
		\begin{itemize}
			\item Grundgesetz, Artikel 4, Absatz 3
			\item Wehrstrafgesetz (WStG), § 16 Fahnenflucht
			\item Zivildienstgesetz (ZDG), § 53 Dienstflucht
			\item Strafgesetzbuch (StGB), § 30 Versuch der Beteiligung
			\item Wehrpflichtgesetz (WPflG), § 3 Inhalt und Dauer der Wehrpflicht
		\end{itemize}
	\end{frame}
	
	\begin{frame}
		\frametitle{Quellenverzeichnis}
		\begin{itemize}
			\item {[ASFRAB\_GS], zuletzt besucht am 2015-03-24:
			\href{http://www.asfrab.de/fileadmin/user_upload/media/pdf/Dokumentation_TKDV_2007_bis_2010.pdf}{http://www.asfrab.de/fileadmin/user\_upload/media/pdf/\\Dokumentation\_TKDV\_2007\_bis\_2010.pdf}}
	
			\item {[ASFRAB\_TV], zuletzt besucht am 2015-03-24: \href{http://www.asfrab.de/wehrpflichtinfos/totalverweigerung.html}{http://www.asfrab.de/wehrpflichtinfos/totalverweigerung.html}} 
			\item {[WIKIPE\_TV], zuletzt besucht am 2015-03-24: \href{https://de.wikipedia.org/wiki/Totalverweigerung}{https://de.wikipedia.org/wiki/Totalverweigerung}}
			\item http://www.thing.de/neid/archiv/6/text/total.htm
		\end{itemize}
	\end{frame}
	  
\end{document}
